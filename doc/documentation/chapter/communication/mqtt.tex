\selectlanguage{german}

\subsection{Einführung in MQTT}
MQTT (Message Queuing Telemetry Transport) ist ein leichtgewichtiges Protokoll, das speziell für die 
Kommunikation zwischen Geräten mit begrenzten Ressourcen entwickelt wurde. Es zeichnet sich durch seine 
Effizienz und geringe Bandbreitennutzung aus, was es ideal für den Einsatz in IoT (Internet of Things) und 
anderen Umgebungen mit eingeschränkten Netzwerkressourcen macht \cite{MQTTWikiedia}.

\subsection{MQTT - Architektur}
MQTT arbeitet nach einem Publisher/Subscriber-Modell, bei dem Clients Nachrichten zu bestimmten
Themen ("Topics") veröffentlichen oder abonnieren können. Ein zentraler Broker verwaltet diese 
Kommunikation, indem er die Nachrichten vom Publisher an die abonnierten Subscriber weiterleitet. 
Es muss also gewährleistet sein das dieser Broker funktioniert und erreichbar ist, denn ohne ihn ist keine
Kommunikation zwischen Subscriber und Publisher möglich. 

\subsection{Sicherheitsaspekt}
Authentifizierung stellt sicher, dass nur autorisierte Geräte Zugang zum Broker erhalten. Es wurde entschieden, 
dass unser Message-Broker nur Kommunikation mit authentifizierten Publishern und Subscribern erlaubt. 
Nicht authentifizierte Clients werden abgelehnt.
Die Verschlüsselung der Kommunikation zwischen Clients und dem Broker erfolgt über TLS (Transport Layer Security).
Dies schützt die Daten vor Abhören und Manipulation durch Dritte.

\subsection{Setup des Message-Brokers}
Um über das MQTT - Protokoll zu kommunizieren, ist der erste Schritt die Installation und Konfiguration 
eines MQTT-Brokers. Bekannte MQTT-Broker sind Mosquitto und HiveMQ. Wir haben uns für den Eclipse - 
Mosquitto - Docker - Container entschieden \cite{eclipse_mosquitto_docker}. Wir konfigurieren den Message Broker
nun über die mosquitto.conf, das nur authentifizierte Clients mit unserem Broker kommunizieren dürfen.
Es wird auch einen Account erstellt, mit dessen Zugangsdaten Clients sich am Broker authentifizieren müssen.

\subsection{Integration mit anderen Komponenten}
Die Integration mit anderen Komponenten unseres Systems ist einfach, denn da MQTT ein weitverbreitetes Protokoll
ist, gibt es in vielen Sprachen Implementationen eines MQTT - Clients, über den Nachrichten an den Message-Broker
und somit andere Clients gesendet werden kann. Es gilt also nur noch festzulegen wie unsere Nachrichten aussehen. 
Da die Nachrichtenformate aber den speziellen Bedingungen und Anforderungen der Kommunikationspartner unterliegen
legen wir uns nicht darauf fest das dieselben Nachrichten-Inhalte im ganzen System auch gleich aussehen.
