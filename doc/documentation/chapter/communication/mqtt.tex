\selectlanguage{german}

\subsection{Einführung in MQTT}
MQTT (Message Queuing Telemetry Transport) ist ein leichtgewichtiges Protokoll, das speziell für die Kommunikation zwischen Geräten mit begrenzten Ressourcen entwickelt wird. Es zeichnet sich durch seine Effizienz und geringe Bandbreitennutzung aus, was es ideal für den Einsatz in IoT (Internet of Things) und anderen Umgebungen mit eingeschränkten Netzwerkressourcen macht.

\subsection{MQTT - Architektur}
MQTT arbeitet nach einem Publisher/Subscriber-Modell, bei dem Clients Nachrichten zu bestimmten Themen ("Topics") veröffentlichen oder abonnieren. Ein zentraler Broker verwaltet diese Kommunikation, indem er die Nachrichten vom Publisher an die abonnierten Subscriber weiterleitet. Es muss gewährleistet sein, dass dieser Broker funktioniert und erreichbar ist, da ohne ihn keine Kommunikation zwischen Subscriber und Publisher möglich ist.

\subsection{Sicherheitsaspekt}
Authentifizierung stellt sicher, dass nur autorisierte Geräte Zugang zum Broker erhalten. Der Message-Broker erlaubt nur Kommunikation mit authentifizierten Publishern und Subscribern. Nicht authentifizierte Clients werden abgelehnt. Die Verschlüsselung der Kommunikation zwischen Clients und dem Broker erfolgt über TLS (Transport Layer Security). Dies schützt die Daten vor Abhören und Manipulation durch Dritte.

\subsection{Setup des Message-Brokers}
Um über das MQTT-Protokoll zu kommunizieren, ist der erste Schritt die Installation und Konfiguration eines MQTT-Brokers. Bekannte MQTT-Broker sind Mosquitto und HiveMQ. Der Eclipse-Mosquitto-Docker-Container wird verwendet. Der Message-Broker wird nun über die mosquitto.conf konfiguriert, sodass nur authentifizierte Clients mit dem Broker kommunizieren dürfen. Ein Account wird erstellt, dessen Zugangsdaten Clients zur Authentifizierung am Broker verwenden müssen.

\subsection{Integration mit anderen Komponenten}
Die Integration mit anderen Komponenten des Systems ist einfach, da MQTT ein weitverbreitetes Protokoll ist und in vielen Sprachen Implementationen eines MQTT-Clients existieren. Über diesen können Nachrichten an den Message-Broker und somit an andere Clients gesendet werden. Es muss nur noch festgelegt werden, wie die Nachrichten aussehen. Da die Nachrichtenformate den speziellen Bedingungen und Anforderungen der Kommunikationspartner unterliegen, wird nicht darauf festgelegt, dass dieselben Nachrichten-Inhalte im ganzen System gleich aussehen.
