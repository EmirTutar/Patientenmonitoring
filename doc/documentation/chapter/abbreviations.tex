\selectlanguage{german}

\section{Grundbegriffe}

\subsection{Docker}
Docker ist eine Open-Source-Softwareplattform, die die Virtualisierung auf Betriebssystemebene nutzt, um Software in Paketen, sogenannten Containern, zu entwickeln, auszuliefern und auszuführen. Container ermöglichen es, eine Anwendung mit allen Teilen, die sie zum Ausführen benötigt, wie Bibliotheken und andere Abhängigkeiten, zu kapseln und auf jedem Linux-System auszuführen, als wären sie auf einem dedizierten Computer installiert. Dies gewährleistet die Konsistenz unabhängig von der Umgebung und verbessert die Sicherheit, da Anwendungen isoliert voneinander ausgeführt werden.

\subsection{MQTT (Message Queuing Telemetry Transport)}
MQTT ist ein Netzwerkprotokoll, das speziell für die Bedürfnisse von Internet-of-Things (IoT)-Anwendungen entwickelt wurde. Es ermöglicht es Geräten, Daten in einem schwach vernetzten Netzwerk effizient zu übertragen. Das Protokoll basiert auf einem Publisher/Subscriber-Modell, das eine hohe Nachrichtenübermittlungseffizienz bei gleichzeitig geringem Ressourcenbedarf bietet. MQTT wird häufig in der Heimautomatisierung, in der Medizintechnik und anderen Bereichen verwendet, in denen eine zuverlässige und effiziente Kommunikation zwischen einer Vielzahl von Geräten erforderlich ist.

\subsection{Raspberry Pi}
Der Raspberry Pi ist ein kleiner, preiswerter Einplatinencomputer, der ursprünglich für Bildungszwecke entwickelt wurde, aber aufgrund seiner Vielseitigkeit und Leistungsfähigkeit in einer Vielzahl von industriellen und Hobby-Projekten eingesetzt wird. Der Pi kann mit verschiedenen Betriebssystemen wie Raspbian, einem Debian-basierten Linux, betrieben werden und unterstützt eine breite Palette von Anwendungen und Programmiersprachen.

\subsection{Fail2Ban}
Fail2Ban ist eine Intrusion Prevention Software, die durch Überwachen von Serverprotokollen auf Muster von Missbrauchsversuchen wie zu viele fehlgeschlagene Anmeldeversuche reagiert. Wenn ein solches Muster erkannt wird, konfiguriert Fail2Ban die Firewall des Hosts temporär so, dass sie IP-Adressen blockiert, von denen aus die verdächtigen Aktivitäten ausgehen, um so das System zu schützen.

\subsection{Ubuntu 22.04 LTS}
Ubuntu 22.04 LTS ist eine Version des Ubuntu-Betriebssystems, die auf Stabilität und langfristige Unterstützung ausgerichtet ist, was sie besonders für Unternehmensumgebungen geeignet macht. Es bietet regelmäßige Sicherheitsupdates für fünf Jahre, was eine zuverlässige Grundlage für Projekte gewährleistet, die eine langfristige Wartbarkeit erfordern.

\subsection{DDclient}
DDclient ist ein Perl-Script, das zur automatischen Aktualisierung von DNS-Einträgen genutzt wird, um dynamisch zugewiesene IP-Adressen mit einem festen Hostnamen zu verknüpfen. Dies ist besonders nützlich in Umgebungen mit dynamischer IP-Zuweisung, wie sie bei vielen Internetdienstanbietern üblich ist.

\subsection{SMTP-Server}
Ein Simple Mail Transfer Protocol (SMTP)-Server ist ein Netzwerkserver, der E-Mails sendet und empfängt. Er spielt eine zentrale Rolle in der elektronischen Kommunikation und wird häufig in Unternehmensumgebungen eingesetzt, um die interne und externe Kommunikation zu verwalten.

\subsection{Matrix}
Matrix ist ein offenes Protokoll für Echtzeitkommunikation, das speziell für die Anforderungen des modernen Internets entwickelt wurde. Es unterstützt verschlüsselte Chat-, VoIP- und Videokonferenzdienste und ermöglicht die Interoperabilität über verschiedene Kommunikationsplattformen hinweg.

