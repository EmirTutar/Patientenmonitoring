\selectlanguage{german}
\subsection{Mailcow}

\subsubsection{Einführung in Mailcow}
Mailcow ist eine Open-Source-Mailserver-Suite, die eine umfassende E-Mail-Lösung bietet. Verschiedene Dienste wie Postfix für den Mail-Transport, Dovecot für die Speicherung und den Zugriff auf E-Mails sowie SOGo als Webmail-Oberfläche werden kombiniert. Eine benutzerfreundliche Weboberfläche ermöglicht die Verwaltung von Domains, Benutzern und E-Mail-Quotas. Durch die Integration von modernen Sicherheitsmechanismen wie DMARC, DKIM und SPF wird ein hohes Maß an Sicherheit für die E-Mail-Kommunikation gewährleistet.

\subsubsection{IMAP / SMTP}
IMAP (Internet Message Access Protocol) und SMTP (Simple Mail Transfer Protocol) sind grundlegende Protokolle für den E-Mail-Verkehr. IMAP wird verwendet, um E-Mails vom Server zu lesen und zu verwalten, während SMTP zum Versenden von E-Mails dient. In Mailcow übernimmt Dovecot die Rolle des IMAP-Servers und Postfix die des SMTP-Servers.

\subsubsection{DNS MX-Record}
Der DNS MX-Record (Mail Exchanger Record) ist ein entscheidender Teil der E-Mail-Infrastruktur. Dieser Record gibt an, welche Mailserver für den Empfang von E-Mails für eine bestimmte Domain verantwortlich sind. Bei der Konfiguration von Mailcow muss der MX-Record der Domain auf den Mailcow-Server zeigen, um sicherzustellen, dass eingehende E-Mails korrekt zugestellt werden. Eine richtige Konfiguration des MX-Records ist notwendig für die Zustellung von E-Mails. Da die richtige Konfiguration des DNS notwendig ist, hat das Mailcow - Dockerized Projekt in ihrer Dokumentation Beschreibungen, wie man diesen richtig aufsetzt \cite{mailcowDockerizedDNS}.

\subsubsection{Sicherheit: DKIM}
DKIM (DomainKeys Identified Mail) ist ein Authentifizierungsprotokoll, das E-Mails mit einer digitalen Signatur versieht. Diese Signatur wird von einem öffentlichen Schlüssel überprüft, der im DNS der sendenden Domain veröffentlicht ist \cite{dkim}. DKIM hilft sicherzustellen, dass E-Mails nicht verändert wurden und wirklich von der angegebenen Domain stammen. In Mailcow kann ein DKIM-Schlüssel-Paar leicht über die Weboberfläche erstellt werden, um die Authentizität der ausgehenden E-Mails zu gewährleisten. Der öffentliche Schlüssel muss dann noch in einem DNS Record eingetragen werden.

\subsubsection{Sicherheit: SPF}
SPF (Sender Policy Framework) Mail Records sind ein Mechanismus zur Verhinderung von E-Mail-Spoofing. Sie ermöglichen es einem Domaininhaber, festzulegen, welche Mail-Server berechtigt sind, E-Mails im Namen ihrer Domain zu senden \cite{spf_erklärung}. Dies wird durch die Veröffentlichung eines SPF-Eintrags im DNS der Domain erreicht, der eine Liste autorisierter IP-Adressen enthält. E-Mail-Empfangsserver überprüfen den SPF-Eintrag der absendenden Domain und verwerfen oder markieren E-Mails als Spam, wenn sie von einem nicht autorisierten Server gesendet wurden. Dadurch wird die Wahrscheinlichkeit reduziert, dass schädliche oder gefälschte E-Mails erfolgreich zugestellt werden.

\subsubsection{Sicherheit: DMARC}
DMARC (Domain-based Message Authentication, Reporting, and Conformance) Mail Records bauen auf den Mechanismen von SPF und DKIM auf, um eine umfassendere E-Mail-Authentifizierung zu bieten. DMARC ermöglicht es, Richtlinien zu definieren, wie E-Mail-Empfangsserver mit Nachrichten umgehen sollen, die SPF- oder DKIM-Prüfungen nicht bestehen \cite{dmarc_erklärung}. Ein DMARC-Eintrag im DNS der Domain legt fest, ob solche E-Mails abgewiesen, in Quarantäne gestellt oder zugestellt werden sollen, und bietet zudem eine Reporting-Funktion. Dadurch können Domaininhaber Berichte über gefälschte E-Mails erhalten und ihre E-Mail-Sicherheitsrichtlinien kontinuierlich verbessern. DMARC hilft somit, die E-Mail-Sicherheit zu stärken und Phishing-Angriffe effektiver zu bekämpfen.

\subsubsection{Installation und Konfiguration}
Zur Konfiguration von Mailcow kann die Dokumentation von Mailcow - Dockerized genutzt werden \cite{mailcowDockerizedDNS}. Dieses Projekt bietet fertige Docker - Container an, die nur noch über eine Konfigurationsdatei angepasst werden müssen. Die Anleitung des Projekts sollte befolgt werden. Es ist wichtig zu wissen, dass die Einstellung des DNS-Servers nicht automatisch erfolgt, sondern manuell erledigt werden muss. Glücklicherweise hilft auch hier die Dokumentation von Mailcow - Dockerized, denn es gibt ein Kapitel, das das DNS-Setup inklusive SPF, DKIM und DMARC erklärt. Es wird auch auf Websites verwiesen, die diese Sicherheitsmechanismen testen können.

\subsubsection{E-Mail-Client}
Zum Versenden von Mails wird ein Mail-Client benötigt, in dem eine E-Mail geschrieben und an den Mail - Server des Accounts gesendet wird. Der Mail - Server kann die E-Mail mit seinem privaten DKIM-Schlüssel signieren und die Mail an den E-Mail-Server des Empfängers weiterleiten. Da das automatische Versenden von Mails ein häufiges Problem darstellt, gibt es hierfür fertige Lösungen. Es wird das Python-Modul ''emails'' verwendet.

