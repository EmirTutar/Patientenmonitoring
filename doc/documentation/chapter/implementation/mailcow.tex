\selectlanguage{german}
\subsection{Mailcow}

\subsubsection{Einführung in Mailcow}
Mailcow ist eine Open-Source-Mailserver-Suite, die eine umfassende E-Mail-Lösung bietet. Sie kombiniert 
verschiedene Dienste wie Postfix für den Mail-Transport, Dovecot für die Speicherung und den Zugriff auf 
E-Mails sowie SOGo als Webmail-Oberfläche. Mailcow bietet eine benutzerfreundliche Weboberfläche zur 
Verwaltung von Domains, Benutzern und E-Mail-Quotas. Aufgrund der Integration von modernen Sicherheitsmechanismen 
wie DMARC, DKIM und SPF gewährleistet Mailcow ein hohes Maß an Sicherheit für die E-Mail-Kommunikation.

\subsubsection{IMAP / SMTP}
IMAP (Internet Message Access Protocol) und SMTP (Simple Mail Transfer Protocol) sind grundlegende Protokolle 
für den E-Mail-Verkehr. IMAP wird verwendet, um E-Mails vom Server zu lesen und zu verwalten, während SMTP zum 
Versenden von E-Mails dient. In Mailcow übernimmt Dovecot die Rolle des IMAP-Servers und Postfix die des 
SMTP-Servers. 

\subsubsection{DNS MX-Record}
Der DNS MX-Record (Mail Exchanger Record) ist ein entscheidender Teil der E-Mail-Infrastruktur. Er gibt an, 
welche Mailserver für den Empfang von E-Mails für eine bestimmte Domain verantwortlich sind. Bei der Konfiguration 
von Mailcow muss der MX-Record der Domain auf den Mailcow-Server zeigen, um sicherzustellen, dass eingehende 
E-Mails korrekt zugestellt werden. Die richtige Konfiguration des MX-Records ist notwendig für die Zustellung 
von E-Mails.

\subsubsection{Installation und Konfiguration}

\subsubsection{Sicherheit: DMAC}

\subsubsection{Sicherheit: DKIM}
DKIM (DomainKeys Identified Mail) ist ein Authentifizierungsprotokoll, das E-Mails mit einer digitalen 
Signatur versieht. Diese Signatur wird von einem öffentlichen Schlüssel überprüft, der im DNS der sendenden
Domain veröffentlicht ist. DKIM hilft sicherzustellen, dass E-Mails nicht verändert wurden und wirklich von
der angegebenen Domain stammen. In Mailcow kann DKIM leicht über die Weboberfläche aktiviert und verwaltet
werden, um die Authentizität der ausgehenden E-Mails zu gewährleisten.

\subsubsection{Sicherheit: SPF}

\subsubsection{E-Mail-Client}

