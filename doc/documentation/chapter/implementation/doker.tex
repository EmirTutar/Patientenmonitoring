\selectlanguage{german}

\subsection{Sicherheit}
Die Sicherheit des Systems wird durch verschiedene Maßnahmen gewährleistet. Dazu gehören die Implementierung von Fail2Ban zur Verhinderung von Brute-Force-Angriffen, die Nutzung von DDclient zur dynamischen DNS-Aktualisierung und die Konfiguration der Uncomplicated Firewall (UFW) zur Verwaltung von Netzwerkanfragen. Diese Maßnahmen tragen dazu bei, das System vor unbefugtem Zugriff und anderen potenziellen Sicherheitsbedrohungen zu schützen.

\subsubsection{Fail2Ban}

Im Rahmen dieses Projekts wird \nameref{subsec:f2b} so konfiguriert, dass bei mehr als fünf fehlgeschlagenen SSH-Anmeldeversuchen die IP-Adresse des Geräts, von dem aus die Anmeldeversuche unternommen werden, dauerhaft gesperrt wird. Diese Sperre bleibt bestehen, bis der Administrator die IP-Adresse manuell wieder entsperrt. Diese strenge Maßnahme dient dazu, die Zugriffskontrolle auf unseren Raspberry Pi zu verstärken und potenzielle Sicherheitsrisiken zu minimieren.

Die Konfigurationsdatei für Fail2Ban (\texttt{jail.local}) wird wie folgt angepasst:

\begin{verbatim}
	[sshd]
	enabled  = true
	port     = ssh
	logpath  = /var/log/auth.log
	maxretry = 5
	bantime  = -1
\end{verbatim}

Diese Einstellungen sorgen dafür, dass Fail2Ban das SSH-Protokoll überwacht und nach fünf fehlgeschlagenen Versuchen die IP-Adresse dauerhaft sperrt. Der Parameter \texttt{bantime = -1} stellt sicher, dass die Sperre unendlich lang andauert, bis der Administrator eingreift.

Durch die Implementierung von Fail2Ban wird die Sicherheit des Systems erheblich verbessert, da automatisierte Brute-Force-Angriffe effektiv abgewehrt werden können.

\subsubsection{DDclient}

DDclient ist ein Perl-basiertes Programm, das dynamische DNS-Dienste (DDNS) unterstützt. Es wird verwendet, um die IP-Adresse eines Systems bei einem DNS-Dienstanbieter zu aktualisieren, wenn sich die IP-Adresse des Systems ändert. Im Rahmen dieses Projekts wird DDclient auf dem Atomic Pi verwendet, der den MQTT-Dienst betreibt. Dadurch wird sichergestellt, dass das Alarmsystem immer unter einem festen Domainnamen erreichbar ist, auch wenn sich die öffentliche IP-Adresse des Netzwerks ändert.Dies ist besonders wichtig für die ständige Kommunikation und Überwachung des Alarmsystem \cite{Ddclient}.


\subsubsection{UFW - Uncomplicated Firewall}

Zur Erhöhung der Sicherheit auf dem Raspberry Pi wird UFW (Uncomplicated Firewall) installiert und konfiguriert. Dabei wird sichergestellt, dass notwendige Dienste erreichbar sind, während andere Verbindungen blockiert werden.\\

Zuerst erfolgt die Installation von UFW, um die grundlegenden Sicherheitsmaßnahmen einzurichten. Die Standardregeln werden so gesetzt, dass eingehender Netzwerkverkehr standardmäßig blockiert wird, es sei denn, es gibt explizite Regeln, die bestimmte Arten von eingehendem Verkehr erlauben. Diese Maßnahme stellt eine grundlegende Sicherheitsmaßnahme dar, um ungewollten oder schädlichen Datenverkehr abzuwehren.\\

Anschließend werden spezifische Verbindungen erlaubt. Zunächst wird der SSH-Zugang freigegeben, um die Fernverwaltung des Systems zu ermöglichen. Dann werden die Ports für HTTP und HTTPS geöffnet, um den Zugang zu Webdiensten zu gestatten.\\

Darüber hinaus werden Docker-spezifische Ports freigeschaltet, um die Kommunikation von Docker-Containern zu ermöglichen. Ebenso werden die MQTT-Ports geöffnet, die für die Kommunikation innerhalb des Systems notwendig sind. Schließlich werden die Netzwerkmanagement-Ports freigegeben, um die grundlegenden Netzwerkdienste zu gewährleisten.\\

Nach der Konfiguration der Regeln wird UFW aktiviert und der Status überprüft, um sicherzustellen, dass die Einstellungen korrekt angewendet wurden. Diese Konfiguration von UFW stellt sicher, dass nur autorisierte Verbindungen zu den notwendigen Diensten auf dem Raspberry Pi zugelassen werden, wodurch die Sicherheit des Systems erhöht wird.


\subsection{Einsatz von Docker}

Für alle Komponenten des verteilten Systems kam \nameref{subsec:docker} zum Einsatz. Diese Technologie wurde gewählt, weil Docker Anwendungen in \nameref{subsec:container}  isoliert, die alle notwendigen Abhängigkeiten und Konfigurationen enthalten. Dadurch konnten die Komponenten auf  den verschiedenen Geräten problemlos betrieben und getestet werden \cite{Aerisdocker}.

\subsubsection{Docker Compose}

Zusätzlich wurde Docker Compose verwendet, um mehrere Container als Teil eines einzigen Anwendungsstapels zu verwalten. In einer YAML-Datei (docker-compose.yml) werden alle benötigten Services, Netzwerke und Volumes definiert. Dies ermöglicht es, alle für einen Service erforderlichen Container mit einem einzigen Befehl zu starten, anstatt jeden Container einzeln konfigurieren und starten zu müssen \cite{DockerCompose}.

\subsubsection{Environment-Datei}

Um sicherzustellen, dass keine vertraulichen Informationen fest im Code verankert werden, wird eine Environment-Datei (.env) verwendet. Diese Datei erleichtert die Verwaltung der Umgebungsvariablen, welche von Docker Compose eingelesen und in den Containern verfügbar gemacht werden \cite{DockerEnv}.
