\selectlanguage{german}

\section{Einleitung}

Die fortschreitende Digitalisierung und technologische Entwicklung bieten große Chancen, die Pflegequalität in medizinischen Einrichtungen zu verbessern. Besonders in der intensiven Patientenüberwachung, wo präzise und zeitnahe Informationen über den Zustand der Patienten lebenswichtig sind, kann der Einsatz moderner Technologie einen entscheidenden Unterschied machen. Dieses Projekt zielt darauf ab, ein Überwachungssystem zu entwickeln, das speziell für die Bedürfnisse der Pflegekräfte und Patienten in Krankenhäusern konzipiert wurde.
\\

In der Praxis wird die Pflegequalität durch die Verfügbarkeit und Genauigkeit der überwachten Daten direkt beeinflusst. Hier setzt unser System an, indem es innovative Technologien nutzt, um eine kontinuierliche, präzise Überwachung zu gewährleisten. Durch die Verwendung von Docker und Raspberry Pi wird eine sichere und isolierte Umgebung geschaffen, in der Anwendungen stabil laufen und leicht skaliert werden können. Dies ist besonders wichtig in der dynamischen Umgebung einer medizinischen Einrichtung, wo Anpassungsfähigkeit und Zuverlässigkeit entscheidend sind.
\\

Ubuntu 22.04 LTS bietet eine solide Grundlage für unser System, da es langfristige Unterstützung und regelmäßige Updates gewährleistet. Zusammen mit der Implementierung von dynamischen DNS-Diensten und Fail2Ban schaffen wir eine sichere Infrastruktur, die den Datenschutz und die Integrität kritischer Patientendaten schützt. Die Integration von Kommunikationstechnologien wie MQTT und Matrix ermöglicht es Pflegekräften, in Echtzeit auf kritische Situationen zu reagieren und verbessert die Koordination innerhalb des Pflegeteams.
\\

Durch den Einsatz dieser Technologien wird nicht nur die Effizienz der Pflegekräfte gesteigert, sondern auch die Sicherheit und das Wohlbefinden der Patienten signifikant verbessert. Dieses Projekt zeigt, wie durch den Einsatz moderner Technologien die Lebensqualität von Menschen verbessert und die Arbeit von medizinischem Personal unterstützt werden kann. Die dokumentierten Abschnitte bieten eine detaillierte Darstellung der technischen Anforderungen, der Architektur sowie der tatsächlichen Implementierung des Systems und motiviert alle Beteiligten, auf diesen Grundlagen weiter aufzubauen und die Technologie zum Wohl der Gesellschaft einzusetzen.

\clearpage
\subsection{Motivation}
Die Überwachung von Patienten in Krankenhäusern ist eine entscheidende Aufgabe, die hohe Anforderungen an Präzision und Zuverlässigkeit stellt. Traditionelle Überwachungsmethoden sind oft personalintensiv und können nicht immer die notwendige kontinuierliche Aufmerksamkeit gewährleisten. Die Integration moderner Technologien wie Videoüberwachung und Computer Vision bietet die Möglichkeit, diese Prozesse zu automatisieren und zu verbessern, um eine kontinuierliche und effektive Überwachung sicherzustellen. Dies ist besonders wichtig in Intensivstationen, wo unvorhersehbare und schnell eintretende Ereignisse rasch erkannt und adressiert werden müssen.

\subsection{Problemstellung}
Trotz des Einsatzes moderner Überwachungstechnologien in Krankenhäusern sind viele Systeme immer noch nicht in der Lage, kritische Situationen wie Stürze, ungewöhnliche Bewegungen oder medizinische Notfälle automatisch zu erkennen. Zudem führen Datenschutzbedenken und die Sorge um die Wahrung der Privatsphäre der Patienten zu zusätzlichen Herausforderungen bei der Implementierung von Überwachungslösungen, die Kameras und andere Überwachungsgeräte nutzen.

\subsection{Zielsetzung}
Das Ziel dieses Projekts ist die Entwicklung eines intelligenten Patienten- überwachungssystems, das mithilfe von Kameras und Computer Vision-Technologien eine kontinuierliche und automatisierte Überwachung bietet. Das System soll in der Lage sein, kritische Situationen wie Stürze oder medizinische Notfälle automatisch zu erkennen und sofort Alarm zu schlagen. Weiterhin soll das System datenschutzkonform gestaltet sein und die Privatsphäre der Patienten respektieren, indem es beispielsweise die Möglichkeit bietet, Überwachungsbereiche gezielt zu anonymisieren oder zu deaktivieren.

