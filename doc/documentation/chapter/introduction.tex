\section{Einleitung}

Im Zeitalter der Digitalisierung und des zunehmenden Einsatzes von kompakten Computermodulen wie dem Raspberry Pi in verschiedenen Anwendungsbereichen, steht die Entwicklung von robusten, skalierbaren und sicheren Systemen mehr denn je im Vordergrund. Dieses Dokument beschreibt ein Projekt, das darauf abzielt, ein Überwachungssystem für die Patientenpflege zu implementieren, wobei der Schwerpunkt auf der Nutzung verschiedener Technologien und Frameworks liegt, um eine effiziente und effektive Lösung zu gewährleisten.

Das Projekt nutzt die Stärken von Docker, um eine isolierte und konsistente Umgebung für die Ausführung verschiedener Dienste auf einem einzigen Raspberry Pi zu schaffen. Dies ist von entscheidender Bedeutung, da es die Wartung und Skalierung des Systems vereinfacht und gleichzeitig die Sicherheit durch die Isolation der einzelnen Anwendungen erhöht. Darüber hinaus stellt die Verwendung von Ubuntu 22.04 LTS eine stabile und unterstützte Basis dar, die die Langlebigkeit und Zuverlässigkeit des Systems sicherstellt.

Die Implementierung dynamischer DNS-Management-Dienste, spezifischer Sicherheitsmechanismen wie Fail2Ban und die Integration fortschrittlicher Kommunikationstechnologien wie MQTT zeigen die Vielseitigkeit und Anpassungsfähigkeit des entwickelten Systems. Zusätzlich zur Kernfunktionalität, die durch die technische Implementierung erreicht wird, umfasst dieses Projekt auch die Erstellung einer angepassten Firewall, die Einrichtung eines Mail-Servers und die Verwendung von Matrix für eine nahtlose und sichere Kommunikation.

Mit diesem Projekt treiben wir nicht nur die technische Entwicklung voran, sondern setzen auch neue Maßstäbe in der Patientenüberwachung. Es zeigt, wie durch den Einsatz moderner Technologien die Lebensqualität von Menschen verbessert und die Arbeit von medizinischem Personal unterstützt werden kann. Die dokumentierten Abschnitte bieten eine detaillierte Darstellung der technischen Anforderungen, der Architektur sowie der tatsächlichen Implementierung des Systems und motiviert alle Beteiligten, auf diesen Grundlagen weiter aufzubauen und die Technologie zum Wohl der Gesellschaft einzusetzen.