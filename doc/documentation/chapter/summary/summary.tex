\clearpage
\section{Zusammenfassung und Ausblick}
Das Projekt zur Patientenüberwachung demonstriert, wie moderne Technologien die Pflegequalität in medizinischen Einrichtungen verbessern können. Durch den Einsatz von Docker, Raspberry Pi und MQTT wurde ein zuverlässiges Überwachungssystem entwickelt, das die Sicherheit und Integrität der Patientendaten gewährleistet.\\

Für die Zukunft gibt es mehrere Erweiterungsmöglichkeiten: Die Erkennung, ob eine Person noch im Bett liegt, könnte helfen auch stürzte auf annonymen Orten wie einer Toilette zu erkennen. Ein Identifikationssystem könnte Pflegekräfte von Patienten unterscheiden und so die Effizienz erhöhen. Zudem könnte die "Bystander Anonymization" den Datenschutz verbessern, indem unbeteiligte Personen in Videoaufnahmen unkenntlich gemacht werden. Schließlich könnte "Digital Fencing" durch virtuelle Grenzen die Bewegungsüberwachung optimieren.\\

Schlussendlich könnte man zudem versuchen, die Genauigkeit des Modells mithilfe einer neueren und größeren Version von YOLO, wie beispielsweise YOLOv8, zu erhöhen. Hierfür müsste man jedoch statt des Raspberry Pi einen externen Server anbinden. Diese Erweiterungen würden das System noch vielseitiger und leistungsfähiger machen und die Sicherheit und das Wohlbefinden der Patienten weiter steigern.\\