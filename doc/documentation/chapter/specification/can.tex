\selectlanguage{german}
\subsection{Kann - Anforderungen}

\subsubsection{MQTT Frontend}
Es kann ein \nameref{subsec:mqtt} - Frontend benutzt werden, um es Entwicklern einfacher zu machen die verschiedenen Nachrichten und Kommunikationskanäle nachzuvollziehen. 

\subsubsection{Gesprochene Information über Patient in Hilfesituation}
In den Minimalanforderungen dieses Systems weiß ein Pfleger nur das ein Patient Hilfe benötigt, nicht aber, welcher der Patienten. Um diese Problematik zu lösen, kann das System anstelle des Alarmtons über eine sprechende Stimme dem Pfleger mitteilen, welcher Patient Hilfe benötigt. 

\subsubsection{Kameraview}
Zusätzlich lässt sich mithilfe des QT-Frameworks eine Kameraview-Anwendung in Python entwickeln \cite{Python} \cite{QT}. Diese Anwendung ermöglicht es Pflegepersonal, die Patienten besser zu überwachen und auch solche Vorfälle zu erkennen, die durch die automatischen Detektoren möglicherweise nicht erfasst werden. Insbesondere kann ein Pfleger mithilfe dieser Kameraview die Extubation eines Patienten frühzeitig bemerken und entsprechend reagieren.